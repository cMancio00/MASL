 \documentclass{beamer}

\usetheme{Singapore}

\usepackage[italian]{babel}
\usepackage{listings}


\title{Parallelizzare una simulazione Monte Carlo con Numba} 
\author{Christian Mancini}
\institute{Università degli studi di Firenze}
\date{\today}

 \begin{document}
 \begin{frame}
    \titlepage
 \end{frame}

\begin{frame}
    \frametitle{Definizioni iniziali}
    \begin{itemize}
        \item Parallelizzare significa portare avanti tante computazioni simultaneamente.
        \item Si possono parallelizzare solo parti di programma indipendenti
        \item Numba ci permette di distribuire la computazione su tutta la CPU e/o GPU
    \end{itemize}
\end{frame}

\begin{frame}[fragile]{Parallelizzazione di una Piccola Simulazione}
    \begin{lstlisting}[language=Python, basicstyle=\ttfamily\scriptsize, keywordstyle=\color{blue}, commentstyle=\color{green!40!black}, numbers=left, numberstyle=\tiny, numbersep=5pt]
    @njit(parallel=True)
    def replicate(sample_size, number_of_simulation,
                  a, b, simulation_results):
        for i in prange(number_of_simulation):
            np.random.seed(111 + i)
            simulated_data = np.random.uniform(
                low=a,
                high=b,
                size=(sample_size,)
                )
            simulation_results[i] = simulated_data.mean()
        return simulation_results.mean()
    \end{lstlisting}
    \end{frame}
    

    








 \begin{frame}

    \begin{thebibliography}{10}
    \bibitem{numba}
    \alert{Numba documentation}
    \newblock{\url{https://numba.readthedocs.io/en/stable/user/jit.html}}

    \end{thebibliography}

    \end{frame}


 \end{document}